\sectionauthor[Romy Schardam]{Conclusion}
Unfortunately, at the time of writing, the prototype is still under construction. A few parts are not finished yet, but it is expected that these changes can be implemented before the end of this project. \\

The journey to get to a flying robot/drone was not one without some setbacks. Between the long ordering times, faulty calculations and SolidWorks and the simulation software not working as expected, the time just seemed to go by too fast. However, the design of the Albatrone was made as best as the team could do under the circumstances.\\

The key aspect of building a bio-mimetic robot, is of course the relation to nature. In our design, this is done by taking inspiration from the albatross. The wing foil, wing shape and wing dimensions were designed to be as close to the actual albatross wings as possible. However, not all dimensions were available online. As a result, the wings are most likely not entirely accurate to the wings of an albatross. Another detail this drone takes from nature is the idea of hollow bones. Birds have hollow bones to make their bones lighter. The parts of the Albatrone are 3D printed and have a infill pattern to be more lightweight. Another possibility was to also take inspiration from the body of the albatross, but unfortunately, this was not in our skillset.\\ 

Of course, not every design choice was based on something in nature. Next to the albatross, inspiration was taken from drones, RC, and commercial planes. These influences can be seen in the body of the drone and in the electronic components. This was done to ensure the Albatrone would be able to be controlled during flight.

In the future, the Albatrone would have an integrated camera to actually be able to help on search and rescue missions as was the goal set in the beginning of this project. On the more ethical side, an integrated camera would put peoples privacy on the line when the drone is misused. To make sure the Albatrone is used for its designed purpose, the Albatrone would not be made available to the general public. Also, to make flying the Albatrone legal, buyers would need to have the right certification. This does create an initial barrier and requires some level of government influence.\\

Depending on if the Albatrone ends up being able to fly, in the future, it might be interesting to look into other materials that can be used to make this drone more lightweight. Also, as mentioned before, changing parts of the design and testing the effect of these changes on the aerodynamic properties using a wind tunnel, would help optimizing the design even more. 


\sectionauthor[Romy Schardam]{Discussion}
The Albatrone team started out as a group of four, but unfortunately ended as a group of three. This increased the workload on the remaining members. Adding the complexity of making a flying robot, this project did not go by without a few doubts in between. However, during this project, different skills were needed. No one from the team had build or designed a flying robot before. This created an initial set-back as first the knowledge needed to build something that could fly needed to be attained. In the end, it is still not sure if the final design works, but having to do the research, it takes a lot to be able to make something fly. The time available for this project, also limited the time that could be spend on acquiring this knowledge. But, some new skills were gained during this project, so even if the prototype is not ready, the experience stays.