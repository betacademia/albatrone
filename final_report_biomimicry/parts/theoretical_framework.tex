\sectionauthor[Simon Blom; Romy Schardam]{Theoretical Framework}
In the theoretical framework, the relevant literature for your project is introduced. This includes not only the literature that is incorporated into the project but also research that was considered but ultimately excluded due to feasibility. The theoretical framework forms the basis for the discussion, helping to contextualize the obtained results. It should include a consideration of the different sources and a justification for why certain literature was included or excluded from the project.

\begin{itemize}
    \item Theoretical Framework: Explores the scientific theories and biological phenomena that form the basis of your design. This section should demonstrate a robust understanding of the biomimicry component and its practical application in your project, supported by relevant scientific literature. It should also include scientific literature to support the non-biological (technical) aspects of your design.
    \item Deliberation of sources: The literature used in your theoretical framework should be evaluated, ultimately leading to a conclusion on which sources to include and which to discard. Describe how and why you adapted these concepts to achieve your prototype's objectives. This provides the foundation for your discussion later in the report.
    \item Relevance of the literature: The literature included in the theoretical framework should have a clear connection to the project. Use sources that are relevant, relatively recent, and appropriate for the project’s scope and context.
    \item Clarity and Structure: The theoretical framework should be logically organized. There are several ways to structure it, but choose a logical approach and maintain consistency.
\end{itemize}
