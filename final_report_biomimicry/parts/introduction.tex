\sectionauthor[Sameh Mikhail]{Introduction}
The official statistics of the United Kingdom's government show that the mean number of missing people in 2022/2023 is around 210 people per month (\cite{uk_search_and_rescue_2023}). In a time and world where mankind has put foot on the moon and where we can travel to the other side of the world, there are people missing in a medium-sized country. Even worse, the United Kingdom is no exception: When searching ``search and rescue missions'' on Google News, the most recent results are a few days and not limiting to a location. 

When people go missing, a search and rescue mission is initiated, involving various entities such as the police, rescue dogs, and volunteers (\cite{ncmec_search_and_rescue}), (\cite{us_doj_search_and_rescue}). However, traditional methods often have limitations, especially when it comes to covering large areas or difficult terrains (\cite{sar_review_2020}). This is where the Albatrone steps in, offering a innovative solution.

The Albatrone is a biomimetic drone modeled after the albatross, which is renowned for its ability to glide effortlessly over vast distances with minimal energy expenditure. Another interesting aspect of the albatross is the fact that it is a heavy bird, but it can still fly. The Albatrone is engineered to remain airborne for extended periods of time, making it a valuable asset to have during search and rescue missions. The key points of the Albatross are
\begin{itemize}
    \item energy efficient, so that it stays for a long period in the air
    \item easy user interface, so that the user does not need to have an intensive training to use it
    \item cheap, so that not only the police, but also non-governmental organisation and smaller organisations can afford it
\end{itemize}
and these should make an impact on search and rescue missions.

\subsection{Abstract}
The Albatrone, a biomimetic drone inspired by the albatross, offers an innovative solution to the challenges of search and rescue missions. The albatross's unique ability to glide long distances with minimal energy consumption serves as the foundation for this drone's design. By mimicking the bird's wing structure, aerodynamics, and energy-efficient flight techniques, the Albatrone is engineered to remain airborne for extended periods. Key features include a lightweight structure with 3D-printed hollow components, inspired by bird bones, and an intuitive user interface to ensure accessibility. Extensive testing, including wind tunnel experiments and material trials, informed the iterative design process, which integrates both biological insights and advanced aeronautical principles. Although the prototype is not fully constructed, the Albatrone demonstrates significant potential to enhance search and rescue capabilities, particularly in large or challenging terrains. Future developments could optimize its design, incorporate integrated cameras, and address ethical concerns regarding privacy to fulfill its mission-driven purpose.